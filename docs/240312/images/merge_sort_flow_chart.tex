\documentclass[border=10pt]{standalone}
\usepackage{tikz}

\usetikzlibrary{shapes,arrows,positioning}

\begin{document}

\tikzstyle{box} = [draw, rounded corners, minimum width=2cm, minimum height=1cm, text centered];

\begin{tikzpicture}[node distance=2cm]

% Start- und Endknoten
\node[box] (start) {Start};
\node[box] (ende) {Ende};

% Teilschritte
\node[box] (merge) {Teile Array in zwei Hälften und sortiere sie rekursiv};
\node[box, below=of merge] (check) {Ist Arraylänge 1 oder 0?};
\node[box, below=of check, yes=ende, no=merge] (ja) {Ja};
\node[box, below=of check, yes=merge2, no=merge] (nein) {Nein};

% Rekursion
\node[box, below=of merge2] (sort2) {Sortiere rechte Hälfte};

% Zusammenführen
\node[box, below=of merge] (merge3) {Führe die beiden sortierten Hälften zusammen};

% Pfeile
\draw[->] (start) -- (merge);
\draw[->] (merge) -- (check);
\draw[->] (check) -- (ja);
\draw[->] (ja) -- (ende);
\draw[->] (check) -- (nein);
\draw[->] (nein) -- (merge2);
\draw[->] (merge2) -- (sort2);
\draw[->] (sort2) -- (merge3);
\draw[->] (merge3) -- (ende);

% Beschriftungen
\node[above=of merge] {Arraylänge $n > 1$};
\node[above=of check] {Arraylänge $n \leq 1$};

\end{tikzpicture}

\end{document}